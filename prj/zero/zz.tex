\documentclass[preprint,preprintnumbers,amsmath,amssymb,showpacs,pre]{revtex4-1}
%\documentclass[preprint,preprintnumbers,amsmath,amssymb]{revtex4-1}
%\documentclass[twocolumn,showkeys,preprintnumbers,amsmath,amssymb]{revtex4-1}
\usepackage{amssymb}
\usepackage{graphicx}
\usepackage[usenames]{color}
\usepackage[normalem]{ulem}
%\usepackage[squaren]{SIunits}

\begin{document}

\title{Analysis of collapse transition of two-dimensional lattice
polymer with partition function zeros}
\author{Ling-Hong Zeng}

\affiliation{Physics Department, Advanced Normal School of Liuzhou, xxxx}

\date{\today}

\begin{abstract}

\end{abstract}

\pacs{87.14.ef,87.15.A-,87.15.ap,87.15.bg}

\maketitle

\section{Introduction}

Polymer is a kind of system with many complex phase transitions, such as
the collapse transition, the glass transition, and so on. These transitions
generally are the results of the interplay between the complex interaction
and conformation entropy, and have attracted a lot of physical interests.

The collapse transition of homopolymers is a typical and fundamental example
of the transitions in polymeric systems. In good solvent regime or at the
high temperature, a homopolymer
would have swollen conformations with large gyration radii. This kind of
state (or phase) has large conformational entropy. Differently, in the poor
solvent regime or at low temperatures, the polymer chain would take compact
conformations. These conformations are generally the results of the
nonspecific attractive
interactions between monomers. The transition between the swollen and the
compact states of homopolymers following the changes of solvent condition
or temperatures is defined as the collapse transition. The transition
condition generally marks the balance of the energy and the entropy of
the homopolymer, and the corresponding transition temperature is historically
called theta temperature, or Flory temperature $T_{\theta}$. The collapse
transition and the related critical exponents have been studied for a
long time with theoretical and computational methods, and has become
the basic problem in polymer physics.

In the computational studies on the collapse transition, the two-dimensional
lattice model with exclusive volume effect is one of the most popular models.
due to its simplicity. Both exact enumerations and Monte Carlo simulations
are used in the related studies. The transition temperature and the
critical exponents related to the collapse transition are evaluated with
this kind of lattice models. However, the results from various calculations
are not consistent with each other quantitatively. For example, the transition
temperature varies from $1.31$ to $1.55$ on the square lattice, and the
crossover exponent $\phi$ is estimated to be in the region between $0.419$
and $0.86$. These differences are apparently larger than the computational
errors.
This may be partially attributed to the finite-size effect or the errors
during the evaluations of the concerned physical quantities. It is expected
to have a precise calculation for the features of the collapse transition.
The analysis of the collapse transition is still a non-trivial work, and
the results and methods would be valuable to understand various transitions
in more complicated polymeric systems (such as proteins).

\textcolor{blue}{exclusive volume effect is not declared here.}

In this work, the transition temperature and the related crossover exponent
are determined based on the analysis on the partition function zeros of
the two-dimensional model on square lattice. In detail, the exact
partition function zeros are determined for the chains with the length
from $12$ up to $27$. Based on the asymptotical trends of the locations of 
the zeros (especially of the first zero), the transition temperature
and the crossover exponent are determined with high precisions. These
results gives out the precise characteristics of the collapse transitions
of two-dimensional lattice polymers, and demonstrates the power of the
method with partition function zeros.

\section{Models and Methods}

\section{Results and Discussions}

\section{Conclusion}

\section*{ACKNOWLEDGMENT}

\begin{thebibliography}{10}

\bibitem{Adermann2004}
K. Adermann, H. John, L. Standker, and W. G. Forssmann.
{\it Curr. Opin. Biotech.} {\bf 15}, 599 (2004).

\end{thebibliography}

\newpage
\vspace{0.5cm}
\parindent 0pt {\large {\bf Fig.1}}
(a) Schematic diagram of the five-bead model. The thick solid lines
represent the covalent and the peptide bonds. The dashed lines
represent the pseudo-bonds which are used to maintain backbone bond
angles, consecutive $C_{\alpha}$ distances, and residue
L-isomerization. (b) Schematic diagram of the hydrogen bond among
backbone. The thick dashed line is the hydrogen bond, The thin
dashed lines are the pseudo-contacts which are introduced to mimic
the collinear structure of group $CO$ and $NH$ in real proteins.

\vspace{0.5cm}
\parindent 0pt {\large {\bf Table I}}
Sequences of the ionic-complementary EAK16-family peptides.

\newpage

\begin{table}
  \centering
  \caption{}
  \begin{tabular}[t]{c|c}
    \hline
     Name & Sequence \\
    \hline
     EAK16-I  & AEAKAEAKAEAKAEAK \\
     EAK16-II & AEAEAKAKAEAEAKAK \\
     EAK16-IV & AEAEAEAEAKAKAKAK \\
    \hline
  \end{tabular}
\end{table}

%\newpage
%\begin{figure*}[htbp]
%\centering
%\includegraphics[width=15cm]{Fig.3.eps}
%\end{figure*}

\end{document}
