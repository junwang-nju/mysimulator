\documentclass[preprint,preprintnumbers,amsmath,amssymb,showpacs,aps,pre]{revtex4-1}
%\documentclass[preprint,preprintnumbers,amsmath,amssymb]{revtex4-1}
%\documentclass[aps,pre,twocolumn,showpacs,preprintnumbers,amsmath,amssymb]{revtex4-1}
%\usepackage{amssymb}
%\usepackage{graphicx}
\usepackage[usenames]{color}
%\usepackage[normalem]{ulem}
%\usepackage[squaren]{SIunits}

\begin{document}

\title{Analysis of collapse transition of two-dimensional lattice
polymer with partition function zeros}
\author{Ling-Hong Zeng}

\affiliation{Physics Department, Advanced Normal School of Liuzhou, xxxx}

\date{\today}

\begin{abstract}

\end{abstract}

\pacs{87.14.ef,87.15.A-,87.15.ap,87.15.bg}

\maketitle

\section{Introduction}

Polymer is a kind of system with many complex phase transitions, such as
the collapse transition, the glass transition, and so on. These transitions
generally are the results of the interplay between the complex interaction
and conformation entropy, and have attracted a lot of physical interests.

The collapse transition of homopolymers is a typical and fundamental example
of the transitions in polymeric systems. In good solvent regime or at the
high temperature, a homopolymer
would have swollen conformations with large gyration radii. This kind of
state (or phase) has large conformational entropy. Differently, in the poor
solvent regime or at low temperatures, the polymer chain would take compact
conformations. These conformations are generally the results of the
nonspecific attractive
interactions between monomers. The transition between the swollen and the
compact states of homopolymers following the changes of solvent condition
or temperatures is defined as the collapse transition. The transition
condition generally marks the balance of the energy and the entropy of
the homopolymer, and the corresponding transition temperature is historically
called theta temperature, or Flory temperature $T_{\theta}$. The collapse
transition and the related critical exponents have been studied for a
long time with theoretical and computational methods, and has become
the basic problem in polymer physics.

In the computational studies on the collapse transition, the two-dimensional
lattice model with exclusive volume effect is one of the most popular models.
due to its simplicity. Both exact enumerations and Monte Carlo simulations
are used in the related studies. The transition temperature and the
critical exponents related to the collapse transition are evaluated with
this kind of lattice models. However, the results from various calculations
are not consistent with each other quantitatively. For example, the transition
temperature varies from $1.31$ to $1.55$ on the square lattice, and the
crossover exponent $\phi$ is estimated to be in the region between $0.419$
and $0.86$. These differences are apparently larger than the computational
errors.
This may be partially attributed to the finite-size effect or the errors
during the evaluations of the concerned physical quantities. It is expected
to have a precise calculation for the features of the collapse transition.
The analysis of the collapse transition is still a non-trivial work, and
the results and methods would be valuable to understand various transitions
in more complicated polymeric systems (such as proteins).

\textcolor{blue}{exclusive volume effect is not declared here.}

In this work, the transition temperature and the related crossover exponent
are determined based on the analysis on the partition function zeros of
the two-dimensional model on square lattice. In detail, the exact
partition function zeros are determined for the chains with the length
from $12$ up to $27$. Based on the asymptotical trends of the locations of 
the zeros (especially of the first zero), the transition temperature
and the crossover exponent are determined with high precisions. These
results gives out the precise characteristics of the collapse transitions
of two-dimensional lattice polymers, and demonstrates the power of the
method with partition function zeros.

\section{Models and Methods}

\subsection{Lattice models of polymers}

The lattice polymer is modeled as a two-dimensional self-avoided chain on
the square lattice. The length is defined as the number $N$ of the monomers in
the chain. For this kind of model of polymers, the position of the
monomer $i$ is ${\mathbf r}_i=(x_i,y_i)$ relative to the presumed origin, where
$x_i$ and $y_i$ are integers. The bonds of chain connect the monomers with
the neighboring indeices and have the length of unit, namely $|{\mathbf r}_{i+1}
-{\mathbf r}_i|=1$. The exclusive volume effect prohibites the overlaps of the
chain, that is, there are no monomers sharing the same coordinates, ${\mathbf r}_i
\neq {\mathbf r}_j$ for $i\neq j$. 

To consider the hydrophobic interaction between the monomers, the spatially
neighboring monomers which are not connected by a bond would have an attractive
interaction with the strength $\epsilon>0$. The Hamiltonian of the polymer thus
takes the form,
\begin{equation}
{\mathcal H}=-\sum_{i<j-1}\epsilon \Delta(|{\mathbf r}_i-{\mathbf r}_j|)=
-\epsilon K \, ,
\end{equation}
in which $\Delta(D)=1$ when $D=1$, and zero otherwise, and the integer $K$
is the number of the contacts between monomers and marks the energy level
that the concerned conformation belongs to. Therefore, the partition
function of the polymer would be written as a polynomial,
\begin{equation}
Z=\sum_{K=0}^{K_{\text{max}}} \Omega(K) s^K \, ,
\end{equation}
where the quantity $s=exp(\epsilon/T)=exp(\beta\epsilon)$, $\Omega(K)$ is the dentisty
of states (DOS) of the system ans records the number 
of conformations on the energy level $K$ and $K_{\text{max}}$ is the
maximum of the index $K$. When the DOS is obtained, the partition function and
the related thermodynamic functions could be determined easily. In this work,
the DOS of the lattice polymers with various lengths are determined with
enumerations.

\subsection{Partition Function Zeros}

The zeros of partition function $Z$ is defined as the roots $\beta=1/T$
for the equation $Z(\beta)=\sum_E n(E)e^{-\beta E}0$ in complex plane. These
roots are declared to be
tightly related to the phase transition of the system described with the
partition function $Z$ by Yang and Lee, and Fisher. This offers a new insight
and method to determine the properties of the phase transitions, and has been
used in critical phenomena, helix-coil transition, and the folding transition of
proteins and heteropolymers. In this work, we apply this method to investigate
the collapse transition of homopolymer systems.

Clearly, when $\beta=\beta_c=1/T_c$ is a root of the equation $Z(\beta)=0$, all
the thermodynamic potentials would be divergent at the temperature $T_c$. When
$T_c$ is a positive real number, this kind of divergence indicates a phase
transition with the discontinuity of thermodynamic potentiaks. For the systems
with finite size, there are no real roots for the equation $Z=0$. The locus of
the zeros would intersect with the real axis at $N\rightarrow\infty$, which
indicates the phase transition at the thermodynamic limit. Therefore, the zeros
close to the real axis and the trends of the zeros to approach to the real axis
would bring us a lot of information about the transitions. This establishes a
method to analysis the phase transition based on the models woth finite sizes.

Close to the transition temperature $T_{\theta}$, the collapse transition would
have scaling behavior. It is easy to derive that the partition function around
the transition temperature also follows the scaling law,
\begin{equation}
\ln Z_N(\tau) \sim N^{-(\alpha+2)\phi} g(\tau N^{\phi}) \,
\end{equation}
in which $\tau=(T-T_{\theta})/T_{\theta}$, and $\alpha$ and $\phi$ are the
concerned critical exponents. The definition of the function $g(x)$ could be
refer to the references. This kind of scaling is rationally assumed to be also
valid in the complex plane. For the equation $Z(\tau)=0$, the root has to
satisfy the relation that $\tau N^{\phi}$ is invariant considering the scaling
behavior. This gives out the trend of the zeros $\tau_c \sim N^{-\phi}$ in the
complex plane when the zeros are close to the real axis. Therefore, the
transtion temperature $T_{\theta}$ and the critical exponent $\phi$ could be
extrapolated based on this relation. This simple relation could help us to find
out the characteristics much more easily.

Besides, based on sophisticated derivation, some other factors could be used
to characterized the phase transition, such as the factors $\gamma$ and
$\alpha$ in the reference \cite{Adermann2004}. The factor $\gamma=d R_T/d I_T$
(where $R_T$ and $I_T$ are the real and image parts of the complex temperature
corresponding to the roots of equation $Z=0$) defines the slopes in the locus
of zeros to cut the real axis, and reflects the strength of the phase
transition. The factor $\alpha$ is related to the density $\rho$ of zeros on the line
toward real axis as the relation $\rho \sim I_T^{\alpha}$. This factor could
be used to judge the order of the phase transition. Clearly, these factors can
be calculated from the zeros close to real axis.

Practically, for a system with only discrete energy levels, the partition
function could be
re-formulated as a polynomial $Z(s)=\sum_J n(J) s^J$ where the variable
$s=exp(-\beta\epsilon)$ and $J=E/\epsilon$. Here $\epsilon$ should be carefully
selected so that the quantity $J=E/\epsilon$ is an integer for all the energy
$E$. For discrete energy levels, $\epsilon$ is generally possible, such as
taking the value of the minimal scale in energy measurements. In this case, the
original equation could be transformed as the problem to solve the polynomial
equation $Z(s)=0$, which would be easy to implement. The precise determination
of the zeros may avoid the fluctuations of physical quantities with other
methods (such as Monte Carlo simulations).

\section{Results and Discussions}

The normalized DOS for the polymers of various lengths ($N=12-31$) are given
in Fig.1(a). It is found that the DOS has a peak with a small number $K$ of
contacts, and then decreases repaidly when $K$ increases. The DOS generally
takes a concave shape in the region with large $K$. This feature ensures the
existence of the collapse transition. The heat capacities at various
temperatures are also calculated (as shown in Fig.1(b)). Folowing the increase
of chain length, the transition becomes more cooperative with higher peaks and
narrower dispersions for the heat capacities. This reflects the trends toward
the thermodynamic limit.

The zeros of the partition functions of the polymers with various lengths are
calculated with PLOY algorithm (using Scilab). Since the coefficients of the
equation $Z(s)=0$ are all real, the roots generally form the conjugate pairs
except the real roots. There are no positive real roots for all partition
functions, which indicates there are strict phase transition for the polymers
with finite sizes. The complex zeros for some represnetative chains are shown
in Fig.2. They are almost aligned in lines. The complex zero with the largest
real component (which is generally positive) is generally approaching toward
the real axis following the increase of the chain length. This zero is related
to the collapse transitions (marked as TZ) and would be investigated with more
attentions. This kind of complex zeros (TZ) for various polymers are collected
(as shown in Fig.3(a)). These points approach to the real axis gradually.
Based on the dicussion in Section ``Model and Method'', these zeros would
have a scaling behavior when close to their limit. For their real and image
components $R_{TZ}$ and $I_{TZ}$, the trends would have the forms,
$R_{TZ}-R_{\theta} \sim N^{-\phi}$ and $I_{TZ} \sim N^{-\phi}$. Here,
$R_{\theta}$ defines the limit of zeros in real axis, which is related to the
transition temperature. Therefore, the critical exponent $\phi$ could be
determined easily by the linear fitting for the log-log plot of the relation
between the image part $I_{TZ}$ of zeros and the chain length $N$ (as shown in
Fig.3(b)). To avoid the deviation of the chains with small sizes, the fitting
is carried out for the chains with their lengths larger than $xx$. The
corresponding exponent $\phi=xx \pm xx$. With this exponent, the transition
temperature $T_{\theta}$ could be determined through the linear fitting for
the relation between the real part $R_{TZ}$ of zeros and the scaled length
$N^{-\phi}$ (as shown in Fig.3(c)). The resultant $R_{\theta}=xx \pm xx$ and
the corresponding transition temperature is $T_{\theta}=xxx \pm xx$.

The slope of the zeros to cutting the real axis is also determined
with the similar procedures.

\section{Conclusion}

\section*{ACKNOWLEDGMENT}

\begin{thebibliography}{10}

\bibitem{Adermann2004}
K. Adermann, H. John, L. Standker, and W. G. Forssmann.
{\it Curr. Opin. Biotech.} {\bf 15}, 599 (2004).

\end{thebibliography}

\newpage
\vspace{0.5cm}
\parindent 0pt {\large {\bf Fig.1}}
(a) Schematic diagram of the five-bead model. The thick solid lines
represent the covalent and the peptide bonds. The dashed lines
represent the pseudo-bonds which are used to maintain backbone bond
angles, consecutive $C_{\alpha}$ distances, and residue
L-isomerization. (b) Schematic diagram of the hydrogen bond among
backbone. The thick dashed line is the hydrogen bond, The thin
dashed lines are the pseudo-contacts which are introduced to mimic
the collinear structure of group $CO$ and $NH$ in real proteins.

\vspace{0.5cm}
\parindent 0pt {\large {\bf Table I}}
Sequences of the ionic-complementary EAK16-family peptides.

\newpage

\begin{table}
  \centering
  \caption{}
  \begin{tabular}[t]{c|c}
    \hline
     Name & Sequence \\
    \hline
     EAK16-I  & AEAKAEAKAEAKAEAK \\
     EAK16-II & AEAEAKAKAEAEAKAK \\
     EAK16-IV & AEAEAEAEAKAKAKAK \\
    \hline
  \end{tabular}
\end{table}

%\newpage
%\begin{figure*}[htbp]
%\centering
%\includegraphics[width=15cm]{Fig.3.eps}
%\end{figure*}

\end{document}
