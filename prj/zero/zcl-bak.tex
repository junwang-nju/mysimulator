%\documentclass[preprint,preprintnumbers,amsmath,amssymb,showpacs,aps,pre]{revtex4-1}
%\documentclass[preprint,preprintnumbers,amsmath,amssymb]{revtex4-1}
\documentclass[aps,pre,twocolumn,showpacs,preprintnumbers,amsmath,amssymb]{revtex4-1}
%\usepackage{amssymb}
%\usepackage{graphicx}
\usepackage[usenames]{color}
%\usepackage[normalem]{ulem}
%\usepackage[squaren]{SIunits}

\begin{document}

\title{Analysis of adsorption transition of two-dimensional lattice
polymer with partition function zeros}
\author{Le Chang, Jun Wang, Wei Wang}

\affiliation{National Laboratory of Solid State Microstructure and
Physics Department, Nanjing University, Nanjing 210093, China
}

\date{\today}

\begin{abstract}
The adsorption transition of two-dimensional lattice polymers are
analyzed with the methods of partition function zeros. Through the
enumeration of all the conformations, the density of states are
and the partition function zeros are evaluated precisely. Based on
the finite-size effect, the critical temperature and crossover exponent
of the transition are determined with extrapolation. The order and the
strength of the transitions of polymers with various lengths are also
calculated. The size dependence of the features of the adsorption
transition is discussed. These results give out a systematical
analysis for the adsorption transition of unbound polymer molecule.
\end{abstract}

\pacs{87.14.ef,87.15.A-,87.15.ap,87.15.bg}

\maketitle

\section{Introduction}

Polymer is a kind of soft condensed-matter system with many complex
phase transitions \cite{LifshitzRMP78,PrivmanBook81,ChengBook08,
deGennesBook79,PhysRep},such as the collapse transition\cite{FloryJCP49,
FloryBOOK67,WilliamsARPC81,OYJCP08}, the glass transition\cite{GibbsJCP58,
JackleRPP86}, and so on. These transitions generally are the results
of the interplay between the complex interaction and conformational
entropy, and have attracted a lot of physical interests
\cite{LifshitzRMP78,FloryJCP49,FloryBOOK67,WilliamsARPC81,GibbsJCP58,
JackleRPP86,deGennesBook79,StephenPLA75,RigByJCP1987,BellRMP93,
ChengARMR,PrivmanBook81,ZivPCCP09,ChengBook08,PCCP10,JCP09,JCP09b,JCP10,
PRL95,PhysRep,OYJCP08,BinderBook95}.

The adsorption transition of a homopolymer through the interaction with
an interface is a typical and fundamental example of phase transitions
in polymeric systems\cite{LifshitzRMP78,deGennesBook79,ChengBook08,PhysRep,
PCCP10, JCP09,JCP09b,JCP10,PRL95,BinderBook95}. It is related to the
competition between translation entropy of the molecule and the adsoption
affinity of the interface. The connectivity of the polymer would also 
affect this kind of processes. With the help of powerful simulation skills,
the complex phase structures related to adsorption are outlined\cite{PCCP10,
JCP09,JCP09b,JCP10,JCP08,Pull09}. Especially, the transition between a short
free polymer and its bound state is observed
to be first-order-like, which is different from the phase-transition
feature in thermodynamic limit with scaling analysis. This kind of finite-size
effect is unusual. Is this phenomenon an artifact related to the selection
of the reaction coordinates in the simulations? What are the physical factors
to produce such a kind of first-order-like transition and the related
conversion of transitions? Would the asymptotic behavior for longer chains
be affected by this kind of unusual behavior of adsorption? The answers to
these questions would not only be helpful to understand the underlying physics
of adsorption transitions of polymers, but also be suggestive to analyse
the adsorptions of various chemical and biological polymers with finite sizes.

In this work, we employ the method of partition function zeros to analyze the
adsorption transitions of a series of two-dimensional model polymers to an
adhesive edge on the square lattice. Full enumerations of the conformations
enables a precise determination of various physical quantities. The partition
function zeros in the complex temperature plane are determined for various
size of chains. The asymptotic behavior of transition temperatures is analyzed
and, thus, the crossover exponent is obtained. The finite-size effect of the
orders and strengths of the transitions is also discussed based on the
information of zeros. The first-order-like feature is confirmed. Based on
the bond correlation, the variation of the features of phase transtions
are qualitatively explained with a two-level model. These results give out a
systematic characterization of the adsorption transition for 2-d
lattice polymers.

%\section{Models and Methods}

%\subsection{Lattice models of polymers}

The lattice polymer is modeled as a two-dimensional self-avoided chain
on the square lattice. This model is popular and simple to study
the physics of polymer systems. The length $L$ is defined as the number of the
monomers in the chain. For this model of polymers, the position
of the monomer $i$ is ${\mathbf r}_i=(x_i,y_i)$ relative to the presumed
origin, where $x_i$ and $y_i$ are integers. The bonds of chain connect
the monomers with the neighboring indices and have the length of unit,
namely $|{\mathbf r}_{i+1}-{\mathbf r}_i|=1$. The exclusive volume effect
prohibits the overlaps of the chain, that is, there are no monomers sharing
the same coordinates, ${\mathbf r}_i\neq {\mathbf r}_j$ for $i\neq j$.
As a simplification, there are no interactions between monomers, which
describes the condition of good solvent.

The adsorption interface is mimicked with an edge in 2-d case. The edge
is located at the position $y_e=-1$. All the monomers cannot penetrate
over the edge, that is, $y_i>y_e$ is required due to the exclusive volume
effect. The monomers neighboring to the edge (namely $y_i=0$) would have
an energy gain $-\epsilon<0$ as the attraction. This kind of interaction
corresponds to the cases with short-range attractions.
Thus, the Hamiltonian of the system thus takes the form,
${\mathcal H}=-\sum_i \epsilon \delta(y_i)\Theta(y_i) =-\epsilon K$,
where the function $\delta(x)=1$ when $x=0$ and is zero otherwise, the
function $\Theta(x)=1$ when $x\ge 0$ and is infinite otherwise, and the
integer $K$ is the number of contacts with the edge , which marks the
energy level of system. Generally, the quantity $\epsilon$ is picked
as the unit of energy, namely $\epsilon=1$.
To keep the system with a finite concentration, a fixed boundary
is located as the position $y_u$. In this work, we take $y_u=30$
which is apparently larger than the size of polymers, and would
not introduce unexpected boundary effects.
It is worth pointing out that the complex arrangement of the polymer
in the interface is suppressed for the 2-d cases, which
avoids the complexity related to the crystallization of polymer on
the interface, and helps us concentrate into the adsorption
transition.

For this kind of interaction, the partition function of the system would be
expressed as a polynomial of $s$, $Z=\sum_{K=0}^{K_{\text{max}}} \Omega(K) s^K$
where the quantity $s=\exp(\epsilon/T)=\exp(\beta\epsilon)$, $\Omega(K)$ is
the density of states (DOS) of the system and records the number
of conformations on the energy level $K$ and $K_{\text{max}}$ is the
maximum of the index $K$. When the DOS is obtained, the partition function
and the related thermodynamic functions could be determined easily and precisely.

In this work, the DOS of the lattice polymers of various sizes are
determined with enumerations. Practically, a conformation is represented
as a string of directions. All the direction strings are enumerated.
Due to the existence of the edge, shift operations along the $y$ direction
are used to generate all possible conformations. Only the conformations
without overlap with the edges
are recorded. It is worth noting the conformations which share a same
direction string after the rotation, reflection and/or translation operations are not
regarded as the same conformation since the boundary eliminates the
symmetry of the system, which is different from the previous lattice
enumerations \cite{CieplakPRL98,PandePRL96}.

%\subsection{Partition Function Zeros}

The zeros of partition function $Z$ are the roots $\beta_i=1/T$
for the equation $Z(\beta)=\sum_E n(E)e^{-\beta E}=0$ in the complex plane.
Here the index $i$ is assigned based on the distances of the zeros away
from the real axis (generally in upper or lower complex plane respectively).
For the systems with discrete levels, the equation $Z(s)=0$ would become
a polynomial equation, and is easy to solve. The expected solutions of
$\beta$ could be obtained with the transformation $\beta=-\ln s/\epsilon$.
These roots are tightly related to the phase transition
of the system described with the partition function $Z$ \cite{LYPR52,FisherBook}.
This method does not depend on the choice of reaction coordinates, and offers
a powerful tool to determine the properties of the phase transitions, and has been
used in many investigations for phase transitions\cite{LipowskiIJMPB05,JCP10b,
HansmannPRL00,HansmannPA01,WangJCP03,ChenPA05}. In this work, this
method is applied in the analysis for the adsorption transition of homopolymer
systems.

Clearly, when $\beta_0=\beta_c=1/T_c$ is a root of the equation $Z(\beta_c)=0$,
all the thermodynamic potentials would be divergent at the temperature $T_c$.
A positive real $T_c$ indicates a phase
transition with the discontinuity of thermodynamic potentials. For the systems
with finite size, there are generally no real roots for the equation $Z=0$.
If the locus of the zeros close to the real axis could intersect with the
real axis when $L\rightarrow\infty$, the system would have a phase transition
in the thermodynamic limit. Therefore, the zeros close to the real axis and
the trends of the zeros to approach to the real axis would bring us a lot of
information about the transitions. For example, based on the scaling analysis,
the zero $\beta_0$ would have the scaling behavior when close to the
transition temperature, $\beta_0-\beta_c\sim N^{-\phi}$. Here, $\phi$ is
the crossover exponent. The analysis of the zeros $\beta_0$ provides a
systematical way to deduce transition temperature and crossover exponent.
Besides, some other factors for the characterization of phase transitions, such as
the factors $\gamma$ and $\alpha$ in the reference \cite{WangJCP03}, could also
be calculated with the zeros. Here, the factor $\gamma=d \beta^R/d \beta^I|_{\beta_0}$
(where $\beta^R$ and $\beta^I$ are the real and image parts of the complex
temperature corresponding to the roots of equation $Z=0$) defines the slopes
in the locus of zeros to cut the real axis, and reflects the strength of the phase
transition. The factor $\alpha$ is related to the density $\rho$ of zeros on
the line toward real axis as the relation $\rho \sim (\beta^I)^{\alpha}$,
and could be used to judge the order of the phase transition.
For discrete zeros, these factors could be estimated with numerical
differentiations as $\gamma=(\beta_1^R-\beta_0^R)/(\beta^I_1-\beta^I_0)$,
$\alpha=(\ln \rho_1-\ln \rho_0)/(\ln\frac{\beta_2^I+\beta_1^I}{2}-
\ln\frac{\beta_1^I+\beta_0^I}{2})$, in which $\rho_i=1/|\beta_{i+1}-\beta_i|$,
and $|x|$ defines the norm
of the complex number $x$. Details could be referred to the reference
\cite{WangJCP03}.  These establish the paradigm to analysis
phase transitions through the complex zeros $\beta$ of partition functions.

%\section{Results and Discussions}

With the enumeration of the conformations, the DOS, $N(K)$, for the polymers
of various lengths ($L=8-24$)
are calculated (as shown in Fig.1). These DOS are normalized so that $N(L)=1$
for the full adsorbed state (which is fully extended and aligned along the edge).
It is found that the DOS decreases roughly in an exponential manner. This
reflects the constraints of the rigid wall to the conformations, which
resembles an effective repulsion. The DOS $N(0)$ is generally the largest.
This is related to the translational entropy of the polymer molecules.
Therefore, $N(0)$ would be larger with little spatial constraints
(namely for the cases with larger $y_u$), while the other parts of $N(K)$
would not be changed by those parameters. Another feature of the DOS
is the shallow valley at $K=1$. This valley introduces a free-energy
gap between free state and the bound state. Therefore, a bimodal
population for the polymer (free or bound) could be expected, which
indicated the first-order-like character of adsorption transition.
However, the ratio of the states in the valley to all the adsorbed states
decreases gradually accompanying with the increase of chain length $L$ (as
shown in the inset of Fig.1).
It implies that the gap becomes shallower for longer chains. Thus,
the first-order-like feature of the transition would become weaker.
For the part $K>1$, the DOS decreases monotonically. This indicates
that the formation or growth of the contacts between the polymer and
the edge is not a sudden (or cooperative) process. This corresponds to
the feature of the second-order adsorption transition in the thermodynamic
limit. Accompanied with the weaker effect of the adsorption initiation
at $K=1$ for longer chains, the transition would behave more and more
second-order-like. This is qualitatively consistent with the previous
studies \cite{PCCP10,JCP09,JCP09b}.

With these DOS, the partition function zeros of the polymers with various lengths are
calculated with PLOY algorithm (using the program Scilab). In these solutions of
$\beta$, there are no positive real roots for all partition functions.
This observation indicates
there are no phase transitions for the polymers with finite sizes. This is
consistent with the physics of the polymer system. The complex zeros $\beta$
for some chains with representative lengths are shown in Fig.2. They are
almost aligned in lines. The complex zero $\beta_0$ with the smallest image
component is generally approaching
toward the real axis following the increase of the chain length. When the
length increases to the infinite (namely $L\rightarrow\infty$), this kind of
zeros would reach the real axis to produce a phase transition, namely the
adsorption transition. The related complex zeros ($\beta_0$) for the
polymers of various lengths are collected (as shown in Fig.3(a)). These
points approach to the real axis gradually following the increase of the
length. Based on the above discussions about partition function zeros, the
real and image components $\beta_0^R$ and $\beta_0^I$, have the trends,
$\beta_0^R - \beta_c  \sim  L^{-\phi}$ and $\beta_0^I  \sim  L^{-\phi}$.
Here,
$\beta_c$ defines the limit of the zero $\beta_0(L)$ in real axis, and represents
the critical temperature of adsorption transition. Thus, the critical crossover exponent
$\phi$ could be determined easily through
the linear fitting for the log-log plot of the relation between the image
part $\beta_0^I$ of zeros and
the chain length $L$ (as shown in Fig.3(b)). It is worth noting that the
scaling would be valid when the length is large enough. To avoid the deviation
of the chains with small sizes, the fitting procedure is carried out for the
chains with their lengths larger than $20$. The corresponding exponent
$\phi=0.50658 \pm 0.00379$. This is consistent with previous theoretical
results\cite{PRL95}.
With this exponent, the collapse transition
temperature $T_c$ could be determined through the linear fitting for the
relation between the real part $\beta_S^R$ of zeros and the scaled length
$L^{-\phi}$ (as shown in Fig.3(c)). The resultant $\beta_c=0.47732 \pm 0.00322$
and the corresponding critical temperature of adsorption transition is
$T_c=2.095 \pm 0.014$. From these asymptotic behaviors, there are no signals
for the conversion of the types of phase transitions.

The factors $\gamma$ and $\alpha$ are also important quantities
to characterize the transitions. They could be determined for
each size of polymer chain. These factors are shown in Fig.4(a) and (b).
The $\gamma$ increases gradually following the increase of the chain
length. This indicates a large span of the transition region, and implies
the weakening of first-order feature in adsorption of polymers for
loner chains. More interestingly, the factor $\alpha$ are all negative.
This reflects that the adsorption transition for the investigated chains
are all first-order-like. This is consistent with previous simulation
results \cite{PCCP10,JCP09,JCP09b}.
It is worth noting that the variation of the factor $\alpha$ is
not monotonically following the increase of the polymer length.
For short chains, the absolute value of the factor $\alpha$ is apparently
smaller than the systematic trend toward the thermodynamic limit.
Following the increase of the
length $L$, the effect of free-energy gap becomes more remarkable. Thus,
the value of $\alpha$ increase. This effect becomes important at $L=11$ to
$13$ for present model polymers. For even long chains, the effect of
free-energy gap is reduced gradually.
This kind of variations of transition feature may be related to the local rigidity
of polymers. As shown in Fig.4(c), there is a local maximum for the bond
correlation $C_{\parallel}(i)$ as the probability of two bonds which are
seperated by $i$ nodes and are both parallel to the adhesive edge. This
indicates the neighboring $3$ mononers are correlated and, thus, may
have the cooperative adsorption with the edge. When the chain is short
enough, the local cooperative binding may sufficiently stabilize the adsorption
of the whole chain, and produce the first-order-like transitions. Differently,
for longer chain, the local interaction is not enough to produce the whole
adsorption. The growth of more adsorbed sites from local binding processes
would be gradual. The finally transition would be in second-order manner.
To further illustrate this picture, we model the adsorption of 2-d polymer with
two-level system (free or adsorbed). The effective strength $\Delta$ to balance the 
entropy difference at the temperature $1/\beta_0$ is calculated. This strength
marks the average interaction to produce adsorption. It is found
that the $\Delta$ is smaller than $3$ for the chains with their lengths smaller than $10$.
Therefore, the local cooperativity could offer sufficient interaction for these short chains
to adsorb on the edge. The transition would be dominated by the binding of local
motifs. For longer chains, the $\Delta$ is gradually larger. The interaction between
the interface and polymer could not produce stable attachment of the whole polymer.
The cooperativity of the transition would decrease when $\Delta$ becomes larger.
These illustrate the basic source of first-order-like feature and the size-dependent
conversion of transitions. Besides, A weak oscillation of $\alpha$
is observed. The similar phenomena are also observed previously\cite{JCP10b},
which is ascribed to the difference of chains with odd or even number of
monomers in lattice space. These results establishes a comprehensive picture for
the size effect
of the adsorption transition of polymers.


%\section{Conclusion}

The method of partition function zeros provides a unique way to analyze various
aspects of the phase transition. In this work, the adsorption transition of
the homopolymer is systematically studied. Accompanied with the knowledge
of the length scaling, the transition temperature, critical crossover
exponent, and the order of the transition are determined. This method is
efficient and precise with no statistical errors. This work demonstrates
the power of the method with partition function zeros. This method could
be further applied to other systems with more complexity.

\section*{ACKNOWLEDGMENT}
This work was supported by National Basic Research Program Grants
(Nos. 2007CB814806), National Natural Science Foundation Grants
(Nos. 10974088, 10834002 and 10774069) and the support from
Jiangsu Province (No. BK2009008).

\begin{thebibliography}{10}

\bibitem{PrivmanBook81}
{\it Finit Size Scaling and Numerical Simulation of Statistical Systems},
Ed. V.Privman (World Scientific, Singapore, 1981).


\bibitem{ChengBook08}
S. Z. D. Cheng, {\it Phase Transitions in Polymers: the role of
metastable states} (Elsevier Publications, Netherlands, 2008).

\bibitem{LifshitzRMP78}
I. M. Lifshitz, A. Yu. Grosberg, and A. R. Khokhlov, Rev. Mod. Phys.
{\bf 50}, 683 (1978).

\bibitem{deGennesBook79}
P.G. de Gennes, {\it Scaling Concepts in Polymer Physics} (Cornell
University Press, Ithaca, 1979).

\bibitem{PhysRep}
R.R. Netz, and D. Andelman, Phys. Rep. {\bf 380}, 1 (2003).

\bibitem{FloryJCP49}
P. J. Flory, J. Chem. Phys. {\bf 17}, 303 (1949).

\bibitem{FloryBOOK67}
P. J. Flory, {\it Principles of Polymer Chemistry} (Cornell University
Press, Ithaca, 167)

\bibitem{WilliamsARPC81}
C. Williams, F. Brochard, and H. L. Frisch, Ann. Rev. Phys. Chem.
{\bf 32}, 433 (1981).

\bibitem{OYJCP08}
J. Zhou, Z.C. Ou-Yang, H. Zhou, J. Chem. Phys. {\bf 128}
124905 (2008).

\bibitem{GibbsJCP58}
J. H. Gibbs and E. A. DiMarzio, J. Chem. Phys. {\bf 28}, 373 (1958).

\bibitem{JackleRPP86}
J. Jackle, Rep. Prog. Phys. {\bf 49}, 171 (1986).

\bibitem{StephenPLA75}
M. J. Stephen, Phys. Lett. A {\bf 53}, 363 (1975).

\bibitem{RigByJCP1987}
D. Rigby, and R.-J. Roe, J. Chem. Phys. {\bf 87}, 7285 (1987).

\bibitem{BellRMP93}
K. DeBell, and T. Lookman, Rev. Mod. Phys. {\bf 65}, 87 (1993).

\bibitem{ZivPCCP09}
G. Ziv, D. Thirumalai, and G. Haran, Phys. Chem. Chem. Phys. {\bf 11},
83 (2009).

\bibitem{ChengARMR}
S. Z. D. Cheng and A. Keller, Annu. Rev. Mat. Sci. {\bf 28}, 533 (1998).



\bibitem{PCCP10}
M. Moddel, W. Janke, M. Bachmann, Phys. Chem. Chem. Phys. {\bf 12},
11548 (2010).

\bibitem{PRL95}
M.T. Batchelor and C.M. Yung, Phys. Rev. Lett. {\bf 74}, 2026 (1995).

\bibitem{JCP09}
T. Chen, L. Wang, X.S. Lin, Y. Liu, H.J. Liang, J. Chem. Phys.
{\bf 130}, 244905 (2009).

\bibitem{JCP09b}
L. Wang, T. Chen, X.S. Lin, Y. Liu, H.J. Liang, J. Chem. Phys.
{\bf 131}, 244902 (2009).

\bibitem{BinderBook95}
K. Binder, {\it Monte Carlo and Molecular Dynamics Simulations in
Polymer Science} (Oxford University Press, New York, 1995).

\bibitem{JCP10}
M.G. Deng, Y. Jiang, H.J. Liang, and J.Z.Y. Chen, J. Chem. Phys.
{\bf 133}, 034902 (2010).

\bibitem{CaparicaCPC09}
A. G. Cunha-Netto, R. Dickman, and A. A. Caparica, Comput. Phys. Comm.
{\bf 180}, 583 (2009).

\bibitem{BinderBook10}
K. Binder and D. W. Heermann, {\it Monte Carlo Simulation in
Statistical Physics: An Introduction} (Springer-Verlag, Berlin, 2010).

\bibitem{BiaoZi}
Y.B. Sheng, W. Wang, and P. Chen, J. Phys. Chem. {\bf 114},
454 (2010).

\bibitem{JCP08}
J. Luettmer-Strathmann, F. Rampf, W. Paul, K. Binder, J. Chem. Phys.
{\bf 128}, 064903 (2008).

\bibitem{Pull09}
A. Milchev, V.G. Rostiashvili, S. Bhattacharya, T.A. Vilgis,
arxiv:0909.2435 (2009).

\bibitem {CieplakPRL98}
M. Cieplak, M. Henkel, J. Karbowski, and J. R. Banavar,
Phys. Rev. Lett. {\bf 80}, 3654 (1998).

\bibitem{PandePRL96}
V. S. Pande, A. Yu Grosberg, C. Joerg, and T. Tanaka,
Phys. Rev. Lett. {\bf 76}, 3987 (1996).

\bibitem{LYPR52}
C. N. Yang, and T. D. Lee, Phys. Rev. {\bf 97}, 404 (1952);
{\bf 87}, 410(1952)

\bibitem{FisherBook}
M. E. Fisher, {\it Lectures in Theoretical Physics} (University of Colorado
Press, Boulder, 1965).

\bibitem{LipowskiIJMPB05}
I. Bena, M. Droz, and A. Lipowski, Int. J. Mod. Phys. B {\bf 19},
4269 (2005).

\bibitem{JCP10b}
J. H. Lee, S. Y. Kim, and J. Lee, J. Chem. Phys. {\bf 133}, 114106 (2010).

\bibitem{HansmannPRL00}
N. A. Alves and U. H. E. Hansmann, Phys. Rev. Lett. {\bf 84}, 1836 (2000)

\bibitem{HansmannPA01}
N. A. Alves and U. H. E. Hansmann, Physica A {\bf 292}, 509 (2001)

\bibitem{WangJCP03}
J. Wang and W. Wang, J. Chem. Phys. {\bf 118}, 2952 (2003).

\bibitem{ChenPA05}
C. N. Chen and C. Y. Lin, Phsyica A {\bf 350}, 45 (2005).

\end{thebibliography}

%\newpage
\vspace{0.5cm}
\parindent 0pt {\large {\bf Fig.1}}
The normalized density of state for the chains with the length $L$ of $12$,
$14$, $16$, $18$, $20$, $22$, and $24$, which are represented with different
signs.

\vspace{0.5cm}
\parindent 0pt {\large {\bf Fig.2}}
The partition function zeros $\beta$ in the complex plane for polymers
with the length $L$ of $12$, $14$, $16$, $18$, $20$, $22$, $24$.

\vspace{0.5cm}
\parindent 0pt {\large {\bf Fig.3}}
(a) The zeros $\beta_S$ with minimal image part for various chains in
complex plane. (b) The length dependence of the image part of the zeros
($\beta_S^I$). The solid line is the fitting of $L^{-\phi}$. The
corresponding fit parameter $\phi$ is given. (c) The length dependence
of the real part of the zeros ($\beta_S^R$). The solid line is
the fitting based on scaling relation. The corresponding fit parameter $\beta_c$ is
given.

\vspace{0.5cm}
\parindent 0pt {\large {\bf Fig.4}}
The length dependence of the factors (a) $\gamma$ and (b) $\alpha$
related to the features of transitions.


%\newpage
%\begin{figure*}[htbp]
%\centering
%\includegraphics[width=15cm]{Fig.3.eps}
%\end{figure*}

\end{document}
