\documentclass[preprint,preprintnumbers,amsmath,amssymb,showpacs,aps,pre]{revtex4-1}
%\documentclass[preprint,preprintnumbers,amsmath,amssymb]{revtex4-1}
%\documentclass[aps,pre,twocolumn,showpacs,preprintnumbers,amsmath,amssymb]{revtex4-1}
%\usepackage{amssymb}
%\usepackage{graphicx}
\usepackage[usenames]{color}
%\usepackage[normalem]{ulem}
%\usepackage[squaren]{SIunits}

\begin{document}

\title{Analysis of collapse transition of two-dimensional lattice
polymer with partition function zeros}
\author{Ling-Hong Zeng}

\affiliation{Physics Department, Liuzhou Teachers College,
Liuzhou 545004}

\date{\today}

\begin{abstract}

\end{abstract}

\pacs{87.14.ef,87.15.A-,87.15.ap,87.15.bg}

\maketitle

\section{Introduction}

Polymer is a kind of system with many complex phase transitions
\cite{LifshitzRMP78}, such as the collapse transition\cite{FloryJCP49,
FloryBOOK67,WilliamsARPC81}, the glass transition\cite{GibbsJCP58,
JackleRPP86}, and so on. These transitions generally are the results
of the interplay between the complex interaction and conformational
entropy, and have attracted a lot of physical interests
\cite{LifshitzRMP78,FloryJCP49,FloryBOOK67,WilliamsARPC81,GibbsJCP58,
JackleRPP86,deGennesBook79,StephenPLA75,RigByJCP1987,BellRMP93,
ChengARMR,PrivmanBook81,ZivPCCP09,ChengBook08}.

Interactions between proteins and various objects are fundamental
in biological and material systems. Through the interactions, the
proteins and objects are associated to realize various structures
and functions. The association processes are dynamic and crucial
for the implementation of functions. What kinetics would the association
process be? and what are the key physical ingredients determining
the processes? This is one of the key questions.

With the specific interactions for the associated conformations,
the binding processes are driven by the interactions which
stabilize the associated conformations. The whole processes could
also be viewed as the diffusion on the energy landscape. On the
other hand, the binding process may be related to the deformation
of the monomeric structure. The processes would be controlled by
these two kinds of processes.

Is the fly-casting process obligatory for binding processes?

The problem is a) the fly-casting processes could increase the 
capture radius. However, the binding speed are not apparently
optimized by .... The rates of the binding processes also vary
largely.

It is sometime said that the intermediate observed during the
binding processes are stabilized by the non-native interactions.
Another view is that the intermediates are stabilized by entropy.
(need some references)

Capture Radius is related to the critical concentration for same
stability. (any correlation?) The rate of the binding could also
be related to the barriers after the capture.

Another problem is that the interaction range could modulate the
whole processes. such as the present examples are all related to
the electrostatic interactions. Besides the strength of the interaction,
how the interaction range modulates the binding behavior is another
problem which is not clear.

To consider the generic feature of the binding processes,
a simplified model for a protein and its interaction is introduced,
in which the structural variations of native structure of the protein
before and after the binding are eliminated. In this case, the
corresponding binding process is frustration-free. This model
enables us to test the validity of the fly-casting processes without
the help of non-native interactions.

In the minimalist model of proteins, three factors are considered to describe
the basic physics related to the native energy, the entropy of denatured
state as well as the folding cooperativity. The protein is modeled as
two beads and a flexible link between beads. The interaction is assumed
between the beads, which mimics the interaction inside the proteins.

With the typical length $L$ as the coordinate for protein, the polymeric
entropy for the denature state could be expressed as the a function of $L$,
\begin{equation}
sssss
\end{equation}
Here we use the gaussian approximation for the proteins. Another term
related to the distance between two beads also included to describe the
density of states of the system.

With the competition of two kinds of ingredients in free energy, two typical
states could be produced, which are dominated by energy and entropy,
respectively.

Another important feature is the cooperativity of the folding. This cannot
be utilized from solely the strength of interaction and the entropic terms.
The rigidity for native state is also necessary to included to produce a
sufficient cooperativity. In the present model, the range of interaction is
used to describe this feature. Typically, the attraction with an exponential
form is used. The characteristic length $X_c$ is used to describe the range
of interaction. For a short-range interaction, the related
native basin would be limited in a small region. The native state and the
unfolded state would be well separated by a free-energy barrier. This gives
out a landscape with two-state cooperativity. Differently, for the cases with
large spatial range for the interaction, the free-energy profile would have a
single minimum. The protein could vary with large fluctuations of the
coordinates (the distance between beads). The dynamics would be in a downhill
style. With this parameter, various kinds of geometries of the landscape could
be presented. This feature could be observed also for other kinds of forms of
attractive interactions. This demonstrates the importance of the interaction
range for the feature of protein folding.  This effect of the interaction range
on the cooperativity of protein is also proved in some off-lattice models.

\textcolor{blue}{[interaction with wall]}

With this kind of monomer, the interaction of protein with other objects
would physically contain similar ingredients, the interaction strength and the
range of interactions. The entropic part is related to the translational
motion which is generic for all the cases. The feature of the interaction
between the protein and objects would be the main content to be discussed in
this work. Physically, these quantities are described with 
the corresponding values for monomer as the units.

(for a typical two-state folder,)
As a typical case, the landscape with the coordinates of the distance of
centre mass $R_{CM}$ and the extension of the monomer $d$ is presented.
Following the approaching of the monomer to the wall, the free energy
on average is decreasing due to the attraction between the protein and
the wall, while the stability of the monomer varies largely. When the
distance $R_{CM}$ is large, the free energy profile along the coordinate
$d$ is weakly affected by the interaction of wall. The protein would
moves between folded and unfolded states with equally probability. When the
protein goes close to the wall, say $R_{CM}=...$, the unfolded state is
stabilized gradually since, in the unfolded state, the attractive part of the
protein may have more changes to be close to the wall. However, the
equilibrium entension of protein in unfolded state varies little. When
approaching to a certain distance, a new minimum emerges at the largest
extension of monomer. That is, a new thermodynamically stable state may exist
in such a situation. This appearance of the new state indicates that the
binding process is helped by the induced-fit mechanism (\textcolor{blue}{why
say help}). Quickly, after the occurence of the new minimum, the minimum
related to the unfolded state is destabilized, and the equilibrium location of
the new minimum decreases linearly and quickly with the $R_{CM}$.

It is interesting to find out that when $R_{CM}$ is smaller than a certain
distance, the protein has only one stable state. The native-like conformation
is also destabilized. Only the minimum with large extension exists. In this
stage, the protein would have a large gyration radius and behaves like a
unfolded state. This state has a large confrmational entropy, but the
rotational motion is greatly limited since the interaction between the wall and
the protein constrains the orientation of the monomer. On the other hand,
there is a large energetic gradient around the present distance of
$R_{CM}$. Therefore, the energy gain for the transition state is larger than
that for the native state. The native state is totally destabilized by this
gradient. With both the energetic and entropic contributions, the free-energy
profile takes the shape with single minimum. This kind of situation would be
persistent for a certain range of distance. The fly-casting phenomenon is
obligatory during this binding processes.

When the protein is close to the
wall, the two-state feature can re-established. However, the extension for the
unfolded state is smaller than that for the free monomer. The energy This could
be attributed to the exclusive effect of the wall. When the rotation of the
protein is limited when adsorbed on the wall, the protein becomes almost a
system on the 2-d space. Therefore, the density of states would change from
the 3-d case to 2-d case. This would decrease the contribution of entropy in
the unfolded state. The two-state feature suggests that the barrier exists for
the formation of native state. The folding would still be an activated
process with a barrier smaller than the free case. Therefore, the rate of
protein binding is also affected by this barrier. Since the barrier height is
lower than that in free state, the temperature dependence is weakened. This
kind of barrier could be observed for DNA-binding case.

\textcolor{red}{is Sasai Model available?}
\textcolor{red}{the mapping between simple model and conceptual models?}

For monomers with various stability, the binding processes are also evaluated.
When the stability is stronger than a certain threshold, the fly-casting
phenomenon is prohibited. The barrier exists between folded state and the
unfolded state. If considering the thermal activation, the existence of the
unfolding kinetics is also possible. The fly-casting would be a kinetic
process. \textcolor{red}{thermal activation?} \testcolor{blue}{The kinetic
equation is proposed.} 

Similar to the case of the monomeric case, the range of interaction would also
affect the binding processes. For a constant strength of interaction, a large
interaction range would decrease the gradient of the potential. For a certain
extension of the monomer, there are no sufficient force to destabilize the
native conformations. Comparing their free-energy profiles, the free energy
corresponding to the unfolded state is generally higher than that for the
native state. Yet, when the protein is far away from the wall, the protein
would affected by the attractive interactions. Differently, when the range of
the protein-wall interaction is smaller, the protein would feel the
interaction only when the chain is rather close to the wall. The strong
gradient of the interaction may deform the protein quickly when the protein
goes around the certain distance. For these two cases, the capture radius for
folded conformation and for the unfolded conformation is related to the
distance of the monomeric extension.

There is another characteristic distance for protein binding with another
small-size object (such as another protein), the capture radius $R_T$
determined by the translation motion. The competition between these distances
would produce some different senarios for binding processes. When the
distance $R_T$ is rather large (related to large interaction range), the
fly-casting would occur after the capture behavior. Therefore, the fly-casting
phenomenon would be suppressed. When the distance of the fly-casting capture
is large than $R_T$, the protein would be seized before encounting the
translation barrier. In this situation, a small entropic penalty related to
the translational motion would be necessary to be compensated. This provides a
mechanism for the speedup of the binding processes. On the phase diagram with
the coordinates of $\epsilon/\Delta$ and $R_T/R_U$, there is a unique region
where the fly-casting could happen. 

For a point-like object,
The competition between the distances between the rigid capture radius and the
fly-casting distance would bring other limits for the existence of fly-casting
mechanism. For the rigid binding, there is a maximum for the free energy.
This peak defines a capture radius, after which the monomer would experience a
downhill free erngy, and would be attracted to the bound state in an
irreversible manner, and when they are far away from the ceter molecule, the
adsorption would be activated with an entropic barrier. When the fly-casting
happens, the critical distance is related to the balance of the attractive
force and the folding barrier. When the distance related to fly-casting is
smaller than the capture distance for rigid binding, there would be little
kinetic benefit from the fly-casting behavior, since that the protein has goes
into the downhill stage. The free-energy profile is plotted as xxx. This kind
of process introduces more ruggedness for the folding close to particle,
rather than to introduce apparent gradient of free energy for capture.
Differently, when the fly-casting happens at the place outside of the rigid
binding, there is a sudden decrease of the free energy which produce a capture
of monomer.
















The collapse transition of homopolymers is a typical and fundamental
example of the transitions in polymeric systems \cite{LifshitzRMP78,
FloryBOOK67,deGennesBook79,ZivPCCP09,ChengBook08}. In good solvent
regime or at the high temperature, a homopolymer would have swollen
conformations with large gyration radii. This kind of state (or phase)
has large conformational entropy. Differently, in the poor solvent
regime or at low temperatures, the polymer chain would take compact
conformations. These conformations are generally the results of the
nonspecific attractive interactions between monomers. The transition
between the swollen and the compact states of homopolymers following
the changes of solvent condition or temperatures is defined as the
collapse transition. The transition condition generally marks the
balance of the energy and the entropy of the homopolymer, and the
corresponding transition temperature is historically called theta
temperature, or Flory temperature $T_{\theta}$\cite{FloryJCP49}.
The collapse transition and the related critical exponents have been
studied for a long time with theoretical and computational methods
\cite{deGennesBook79,PrivmanBook81,ZivPCCP09,SaleurJSP86,
DuplantierPRL87,StanleyPRB89,BinderBook95,OYJCP08,CaparicaCPC09,
BinderBook10}, and has become the basic problem in polymer physics
\cite{LifshitzRMP78,deGennesBook79,ZivPCCP09,ChengBook08}.

In the computational studies on the collapse transition, the
two-dimensional lattice model with exclusive volume effect is
one of the most popular models due to its simplicity. Both exact
enumerations and Monte Carlo simulations are used in the related
studies\cite{SaleurJSP86,DuplantierPRL87,StanleyPRB89,OYJCP08,
DerridaJPA85,SenoJP88,ChangPRE93,HeegerJP95,MurthyPRE01}. The
transition temperature and the critical exponents related to the
collapse transition are evaluated with this kind of lattice models.
However, the results from various calculations are diverse
quantitatively. For example, the transition temperature varies from
$1.31$ to $1.55$ on the square lattice, and the crossover exponent
$\phi$ is estimated to be in the region between $0.419$ and $0.86$.
These differences are apparently larger than the computational
errors. This may be partially attributed to the finite-size effect
and the errors during the evaluations of the concerned physical
quantities. It is expected to have a precise calculation for the
features of the collapse transition. The analysis of the collapse
transition is still a non-trivial work, and the results and methods
would be valuable to understand various transitions in more
complicated polymeric systems (such as proteins\cite{ZivPCCP09}).

In this work, the order of the transition, the transition temperature
and the related crossover exponent are determined based on the analysis
using the partition function zeros of the two-dimensional models
on square lattice. In detail, the exact partition function zeros are
determined for the chains with the length from $12$ up to $31$. Based
on the asymptotical trends of the locations of the zeros (especially
of the zeros around real axis), the transition temperature and the
crossover exponent are determined with high precisions. These results
gives out the precise characteristics of the collapse transitions of
two-dimensional lattice polymers, and demonstrates the power of the
method with partition function zeros.

\section{Models and Methods}

\subsection{Lattice models of polymers}

The lattice polymer is modeled as a two-dimensional self-avoided chain
on the square lattice. The length $L$ is defined as the number of the
monomers in the chain. For this kind of model of polymers, the position
of the monomer $i$ is ${\mathbf r}_i=(x_i,y_i)$ relative to the presumed
origin, where $x_i$ and $y_i$ are integers. The bonds of chain connect
the monomers with the neighboring indeices and have the length of unit,
namely $|{\mathbf r}_{i+1}-{\mathbf r}_i|=1$. The exclusive volume effect
prohibites the overlaps of the chain, that is, there are no monomers sharing
the same coordinates, ${\mathbf r}_i\neq {\mathbf r}_j$ for $i\neq j$. 

To consider the hydrophobic interaction between the monomers, the spatially
neighboring monomers which are not connected by a bond would have an
attractive interaction with the strength $\epsilon>0$. The Hamiltonian of
the polymer thus takes the form,
\begin{equation}
{\mathcal H}=-\sum_{i<j-1}\epsilon \Delta(|{\mathbf r}_i-{\mathbf r}_j|)=
-\epsilon K \, ,
\end{equation}
in which $\Delta(D)=1$ when $D=1$, and zero otherwise, and the integer $K$
is the number of the contacts between monomers and marks the energy level
that the concerned conformation belongs to. The quantity $\epsilon$ here
represents the strength of the contact interaction, which is generally taken
as the unit of the energy (namely $\epsilon=1$. Therefore, the partition
function of the polymer would be written as a polynomial,
\begin{equation}
Z=\sum_{K=0}^{K_{\text{max}}} \Omega(K) s^K \, ,
\end{equation}
where the quantity $s=exp(\epsilon/T)=exp(\beta\epsilon)$, $\Omega(K)$ is
the dentisty of states (DOS) of the system ans records the number 
of conformations on the energy level $K$ and $K_{\text{max}}$ is the
maximum of the index $K$. When the DOS is obtained, the partition function
and the related thermodynamic functions could be determined easily.

In this work, the DOS of the lattice polymers with various lengths are
determined with enumerations. Practically, a conformation is represented
as a string of directions. All the direction strings are enumerated. It is
worth noting the conformations which share a same direction string after
rotation and/or reflection operations are regarded as the same conformation.
This kind of operation has been widely used in lattice enumerations
\cite{CieplakPRL98,PandePRL96}.

\subsection{Partition Function Zeros}

The zeros of partition function $Z$ are defined as the roots $\beta=1/T$
for the equation $Z(\beta)=\sum_E n(E)e^{-\beta E}$ in the complex plane.
These roots are declared to be tightly related to the phase transition
of the system described with the partition function $Z$ by Yang and Lee
\cite{LYPR52}, and Fisher\cite{FisherBook}. This offers a powerful method
to determine the properties of the phase transitions, and has been used
in critical phenomena\cite{LipowskiIJMPB05}, the helix-coil transition
\cite{HansmannPRL00,HansmannPA01}, and the folding transition of
proteins and heteropolymers\cite{WangJCP03,ChenPA05}.
In this work, we apply this method to investigate
the collapse transition of homopolymer systems.

Clearly, when $\beta=\beta_c=1/T_c$ is a root of the equation $Z(\beta)=0$, all
the thermodynamic potentials would be divergent at the temperature $T_c$. When
$T_c$ is a positive real number, this kind of divergence indicates a phase
transition with the discontinuity of thermodynamic potentiaks. For the systems
with finite size, there are generally no real roots for the equation $Z=0$.
If the locus of the zeros close to the real axis could intersect with the
real axis when $N\rightarrow\infty$, the system would have a phase transition
at the thermodynamic limit. Therefore, the zeros close to the real axis and
the trends of the zeros to approach to the real axis would bring us a lot of
information about the transitions. This establishes a paradigm to analysis
the phase transition based on the models woth finite sizes.

Close to the transition temperature $T_{\theta}$, the collapse transition
would have scaling behavior. It is easy to derive that the partition
function around the transition temperature also follows the scaling law,
\begin{equation}
\ln Z_N(\beta) \sim N^{-(\alpha+2)\phi} g(\tau N^{\phi}) \,
\end{equation}
in which $\tau=(\beta-\beta_{\theta})/\beta_{\theta}$, and $\alpha$ and
$\phi$ are the concerned critical exponents. The definition of the function
$g(x)$ could be refer to the references\cite{ChangPRE93}. This kind of
scaling is rationally assumed to be also valid around the real axis in
the complex plane. For the equation $Z(\beta)=0$, the root has to satisfy
the relation that $\tau N^{\phi}$ is invariant considering the scaling
behavior. This gives out the trend of the zeros as 
$\beta-\beta_c \sim N^{-\phi}$ in the complex plane when the zeros are
close to the real axis. Therefore, the
transtion temperature $T_{\theta}$ and the critical exponent $\phi$ could be
extrapolated based on this relation. This simple relation could help us to
find out the characteristics of phase transition at the thermodynamic limit
much more easily.

Besides, based on sophisticated derivation, some other factors could be used
to characterized the phase transition, such as the factors $\gamma$ and
$\alpha$ in the reference \cite{WangJCP03}. The factor $\gamma=d R_T/d I_T$
(where $R_T$ and $I_T$ are the real and image parts of the complex temperature
corresponding to the roots of equation $Z=0$) defines the slopes in the locus
of zeros to cut the real axis, and reflects the strength of the phase
transition. The factor $\alpha$ is related to the density $\rho$ of zeros on
the line toward real axis as the relation $\rho \sim I_T^{\alpha}$. This
factor could be used to judge the order of the phase transition. Clearly,
these factors can be calculated from the zeros close to real axis.

Practically, for a system with only discrete energy levels, the partition
function could be
re-formulated as a polynomial $Z(s)=\sum_J n(J) s^J$ where the variable
$s=exp(-\beta\epsilon)$ and $J=E/\epsilon$. Here $\epsilon$ should be carefully
selected so that the quantity $J=E/\epsilon$ is an integer for all the energy
$E$. For discrete energy levels, $\epsilon$ is generally possible, such as
taking the value of the minimal scale in energy measurements. In this case, the
original equation could be transformed as the problem to solve the polynomial
equation $Z(s)=0$, which would be easy to implement. The resultant solutions
of $\beta$ could be obtained with the transformation $\beta=-\ln s/\epsilon$.
The precise determination of the zeros may avoid the fluctuations of physical
quantities with other methods (such as Monte Carlo simulations).

\section{Results and Discussions}

The normalized DOS, $P(E)=N(E)/\sum_E N(E)$, for the polymers of various
lengths ($L=12-31$) are given in Fig.1. The total
numbers $N=\sum_E N(E)$ for the chains with various lengths are listed
in the Table I. It is found that the DOS has a peak with a small number $K$ of
contacts, and then decreases rapidly when the contact number $K$ increases.
The DOS generally takes a concave shape in the region with a large $K$. This
feature ensures the existence of the collapse transition. It is worth noting
that the difference of $K$ between the most conpact conformations (with the
largest $K$) and the most probable conformations (often swollen with the $K$
around the peak of $P(E)$) becomes larger accompanying with the increase of
chain length $N$. The increase of the energetic difference between the
collapsed and the swollen conformations suggests the enhancement of the
cooperativity for longer chains. This indicates that there is a phase
transition when the length of the chain approaches to infinite, and reflects
the trend toward thermodynamic limit.

The zeros of the partition functions of the polymers with various lengths are
calculated with PLOY algorithm (using Scilab). Since the coefficients of the
equation $Z(s)=0$ are all real, the roots generally form the conjugate pairs
except the real roots. In these solutions of $\beta$, there are no positive
real roots for all partition functions. This kind of observation indicates
there are no phase transitions for the polymers with finite sizes. This is
consistent with the physics of the polymer system. The complex zeros $\beta$
for some chains with representative lengths are shown in Fig.2. They are
almost aligned in lines. The complex zero $\beta_S$ with the smallest image
component (namely the zero closest to the real axis) is generally approaching
toward the real axis following the increase of the chain length. When the
length increases to the infinite (namely $L\rightarrow\infty$), this kind of
zeros would reach the real axis to produce a phase transition. Physically,
this transition would be the collapse transition of the polymer system. The
line of $\beta_S(L)$ contains a lot of information about the phase transition.
Presently, the line $\beta_S(L)$ is estimated with the calculated $\beta_S$
for some chains of finite size. The related complex zeros ($\beta_S$) for the
polymers with various lengths are collected (as shown in Fig.3(a)). These
points approach to the real axis gradually following the increase of the
length. Based on the dicussion in Section ``Model and Method'', these zeros
would have a scaling behavior when they are close to their limit. For their
real and image components $\beta_S^R$ and $\beta_S^I$, the trends would have
the forms,
\begin{eqnarray}
\beta_S^R - \beta_{\theta}  &\sim&  L^{-\phi} \, , \\
\beta_S^I  &\sim&  L^{-\phi} \, ,
\end{eqnarray}
Here,
$\beta_{\theta}$ defines the limit of the zero $\beta(L)$ in real axis, which
is related to the collapse transition temperature. Therefore, the critical
exponent $\phi$ could be determined easily through the linear fitting for the
log-log plot of the relation between the image part $\beta_S^I$ of zeros and
the chain length $L$ (as shown in Fig.3(b)). It is worth noting that the
scaling would be valid when the length is large enough. To avoid the deviation
of the chains with small sizes, the fitting procedure is carried out for the
chains with their lengths larger than $20$. The corresponding exponent
$\phi=0.567 \pm 0.014$. With this exponent, the collapse transition temperature
$T_{\theta}$ could be determined through the linear fitting for the relation
between the real part $\beta_S^R$ of zeros and the scaled length $L^{-\phi}$
(as shown in Fig.3(c)). The resultant $\beta_{\theta}=xx \pm xx$ and the
corresponding collapse transition temperature is $T_{\theta}=xxx \pm xx$.

The exponent $\alpha$ is another important quantity related to the transition.
It could also be determined with the similar procedure to that for the
exponent $\phi$. This exponent is related to the local density of zeros close
to real axis, $\rho_S$. Practically, this density is estimated with a
three-point approximation,
\begin{equation}
\rho_S=\frac{2}{|\beta_0-\beta_1|+|\beta_1-\beta_2|} \, ,
\end{equation}
in which $|x|$ gives the norm of the complex number $x$, and the factors
$\beta_0$, $\beta_1$, and $\beta_2$ are indexed with increasing image parts of
the $\beta$. Clearly, the factor $\rho_S$ here is an estimation for the desity
of zeros around real axis based on the discrete zeros. When the length of the
polymer increases approaches to infinity, this quantity would converge to
the expect density related to the thermodynamic limit. Based on the analysis
in the section ``Model and Method'', the density $\rho_S$ would also have a
scaling behavior when the system is close to the thermodynamic limit. The
exponent could be obtained by the fitting procedure similar to the above
calculations. Practically, the fitting is carried out for the chains $N\ge
20$. The resultant exponent $\alpha=xxx$. Clearly, this exponent is obviously
larger than zero. This result indicates that this transition is a second-order
phase transition. This is consistent with the conclusions obtained from other
studies.

Previously, the transition temperature of lattice polymers on square lattice
was estimated as $1.31$\cite{JP82}, $1.55$\cite{BirshteinPolymer85} or
$1.505$\cite{OYJCP08} with Monte Carlo simulations. Their maximal lengths of
simulated polymers are $1.60$, $200$, and $300$. Similar to the calculations
of the transition temperature, the critical exponent $\phi$ is also diverse in
varius works, such as $0.6$\cite{BirshteinPolymer85} with Monte Carlo
simulations, $0.48$\cite{SaleurJSP86} with transfer matrix, and
$0.435$\cite{HeegerJP95} with growth method. Our result is in between
these results. Clearly, our results have no numerical errors for the zeros
related to the polymers with various lengths. It is reasonable that there are
statistical errors in our results. More importantly, the image parts of zeros
help us to determine the exponent more precisely since there are no shift
parameters during fitting. Differently, the processing of the simlation
results related to Monte Carlo simulations may have some artifacts in
parameter fitting. Therefore, our results would probably match the
case in thermodynamic limit. This also demonstrates the power of the method
with partition function zeros.

\section{Conclusion}

The method partition function zeros provides a unique way to analyze various
aspects of the phase transition. In this work, the collpase transition of the
homopolymer is systematically studied. Accompanied with the knowledge of the
length scaling, the transition temperature, critical exponent, and the order
of the transition at the thermodynamic limit are determined. Comparing with
previous results, this method is efficient and precise with no statistical
errors. This work demonstrates the power of the method with partition function
zeros. This method could be further applied to other systems with more
complexity.

\section*{ACKNOWLEDGMENT}

\begin{thebibliography}{10}

\bibitem{LifshitzRMP78}
I. M. Lifshitz, A. Yu. Grosberg, and A. R. Khokhlov, Rev. Mod. Phys.
{\bf 50}, 683 (1978).

\bibitem{FloryJCP49}
P. J. Flory, J. Chem. Phys. {\bf 17}, 303 (1949).

\bibitem{FloryBOOK67}
P. J. Flory, {\it Principles of Polymer Chemistry} (Cornell University
Press, Ithaca, 167)

\bibitem{WilliamsARPC81}
C. Williams, F. Brochard, and H. L. Frisch, Ann. Rev. Phys. Chem.
{\bf 32}, 433 (1981).

\bibitem{GibbsJCP58}
J. H. Gibbs and E. A. DiMarzio, J. Chem. Phys. {\bf 28}, 373 (1958).

\bibitem{JackleRPP86}
J. Jackle, Rep. Prog. Phys. {\bf 49}, 171 (1986).

\bibitem{StephenPLA75}
M. J. Stephen, Phys. Lett. A {\bf 53}, 363 (1975).

\bibitem{RigByJCP1987}
D. Rigby, and R.-J. Roe, J. Chem. Phys. {\bf 87}, 7285 (1987).

\bibitem{BellRMP93}
K. DeBell, and T. Lookman, Rev. Mod. Phys. {\bf 65}, 87 (1993).

\bibitem{ChengARMR}
S. Z. D. Cheng and A. Keller, Ann. Rev. Mat. Sci. {\bf 28}, 533 (1998).

\bibitem{ChengBook08}
S. Z. D. Cheng, {\it Phase Transitions in Polymers: the role of
metastable states} (Elsevier Publications, Netherlands, 2008).

\bibitem{deGennesBook79}
P.G. de Gennes, {\it Scaling Concepts in Polymer Physics} (Cornell
University Press, Ithaca, 1979).

\bibitem{ZivPCCP09}
G. Ziv, D. Thirumalai, and G. Haran, Phys. Chem. Chem. Phys. {\bf 11},
83 (2009).

\bibitem{PrivmanBook81}
{\it Finit Size Scaling and Numerical Simulation of Statistical Systems},
Ed. V.Privman (World Scientific, Singapore, 1981).

\bibitem{SaleurJSP86}
H. Saleur, J. Stat. Phys. {\bf 45}, 419 (1986).

\bibitem{DuplantierPRL87}
B. Duplantier and H. Saleur, Phys. Rev. Lett. {\bf 59} 539 (1987).

\bibitem{StanleyPRB89}
P.H. Poole, A. Coniglio, N. Jan, and H. E. Stanley, Phys. Rev. B
{\bf 39} 495 (1989).

\bibitem{BinderBook95}
K. Binder, {\it Monte Carlo and Molecular Dynamics Simulations in
Polymer Science} (Oxford University Press, New York, 1995).

\bibitem{OYJCP08}
J. Zhou, Z.C. Ou-Yang, and H. Zhou, J. Chem. Phys. {\bf 128}
124905 (2008).

\bibitem{CaparicaCPC09}
A. G. Cunha-Netto, R. Dickman, and A. A. Caparica, Comput. Phys. Comm.
{\bf 180}, 583 (2009).

\bibitem{BinderBook10}
K. Binder and D. W. Heermann, {\it Monte Carlo Simulation in
Statistical Physics: An Introduction} (Springer-Verlag, Berlin, 2010).

\bibitem{DerridaJPA85}
B. Derrida, and H. Saleur, J. Phys. A {\bf 18}, L1075 (1985).

\bibitem{SenoJP88}
F. Seno, and A. L. Stella, J. Phys France {\bf 49}, 739 (1988).

\bibitem{ChangPRE93}
I. Chang and H. Meirovitch, PHys. Rev. E {\bf 48}, 3656 (1993).

\bibitem{HeegerJP95}
P. Grassberger and R. Heeger, J. Phys. I {\bf 5} 597 (1995).

\bibitem{MurthyPRE01}
S. L. Narasimhan, P.S. Krishna, K. P. N. Murthy, and M. Ramanadham,
Phys. Rev. E {\bf 65}, 010801(R) (2001).

\bibitem {CieplakPRL98}
M. Cieplak, M. Henkel, J. Karbowski, and J. R. Banavar,
Phys. Rev. Lett. {\bf 80}, 3654 (1998).

\bibitem{PandePRL96}
V. S. Pande, A. Yu Grosberg, C. Joerg, and T. Tanaka,
Phys. Rev. Lett. {\bf 76}, 3987 (1996).

\bibitem{LYPR52}
C. N. Yang, and T. D. Lee, Phys. Rev. {\bf 97}, 404 (1952); 
{\bf 87}, 410(1952)

\bibitem{FisherBook}
M. E. Fisher, {\it Lectures in Theoretical Physics} (University of Colorado
Press, Boulder, 1965).

\bibitem{LipowskiIJMPB05}
I. Bena, M. Droz, and A. Lipowski, Int. J. Mod. Phys. B {\bf 19},
4269 (2005).

\bibitem{HansmannPRL00}
N. A. Alves and U. H. E. Hansmann, Phys. Rev. Lett. {\bf 84}, 1836 (2000)

\bibitem{HansmannPA01}
N. A. Alves and U. H. E. Hansmann, Physica A {\bf 292}, 509 (2001)

\bibitem{WangJCP03}
J. Wang and W. Wang, J. Chem. Phys. {\bf 118}, 2952 (2003).

\bibitem{ChenPA05}
C. N. Chen and C. Y. Lin, Phsyica A {\bf 350}, 45 (2005).

\bibitem{JP82}
A. Baumgartner, J. Phys. {\bf 43}, 1407 (1982).

\bibitem{BirshteinPolymer85}
T. M. Birshtein, S. V. Buldyrev, and A. M. Elyashevitch, Polymer {\bf 26},
1814 (1985).

\end{thebibliography}

\newpage
\vspace{0.5cm}
\parindent 0pt {\large {\bf Fig.1}}
(a) Schematic diagram of the five-bead model. The thick solid lines
represent the covalent and the peptide bonds. The dashed lines
represent the pseudo-bonds which are used to maintain backbone bond
angles, consecutive $C_{\alpha}$ distances, and residue
L-isomerization. (b) Schematic diagram of the hydrogen bond among
backbone. The thick dashed line is the hydrogen bond, The thin
dashed lines are the pseudo-contacts which are introduced to mimic
the collinear structure of group $CO$ and $NH$ in real proteins.

\vspace{0.5cm}
\parindent 0pt {\large {\bf Table I}}
Sequences of the ionic-complementary EAK16-family peptides.

\newpage

\begin{table}
  \centering
  \begin{tabular}[t]{cc|cc|cc}
    \toprule
     L & N  & L & N & L & N \\
    \hline
     12 & 15037 &
     13 & 40617 &
     14 & 110188 \\
     15 & 296806 &
     16 & 802075 &
     17 & 2155667 \\
     18 & 5808335 &
     19 & 15582342 &
     20 & 41889578 \\
     21 & 112212146 &
     22 & 301100754 &
     23 & 805570061 \\
     24 & 2158326727 &
     25 & 5768299665 &
     26 & 15435169364 \\
     27 & 41214098278 &
     28 & 110164686454 &
     29 & 293922322781 \\
     30 & 784924528667 &
     31 & 2092744741919 &
        & \\
    \toprule
  \end{tabular}
\end{table}

%\newpage
%\begin{figure*}[htbp]
%\centering
%\includegraphics[width=15cm]{Fig.3.eps}
%\end{figure*}

\end{document}
